\documentclass[a4paper,12pt]{book}
\usepackage[utf8]{inputenc}
\usepackage[belarusian]{babel}
\usepackage{amsfonts}
\usepackage{amsmath}
\usepackage{amsthm}

\usepackage{tikz-cd}
\usetikzlibrary{babel}

\addto\captionsbelarusian{\renewcommand{\chaptername}{Разьдзел}}

\newtheorem{theorem}{Тэарэма}[section]
\newtheorem{claim}{Сьцьверджаньне}[section]
\newtheorem{example}{Прыклад}[section]
\newtheorem{definition}{Азначэньне}[section]
\newtheorem{axiom}{Аксыёма}[section]
%\usepackage[ntheorem]

\newcommand\restr[2]{{% we make the whole thing an ordinary symbol
    \left.\kern-\nulldelimiterspace % automatically resize the bar with \right
    #1 % the function
    \vphantom{\big|} % pretend it's a little taller at normal size
    \right|_{#2} % this is the delimiter
}}

\addto\captionsbelarusian{
  \renewcommand{\contentsname}
    {Зьмест}
}

\title{Тэорыя катэгорый: кароткія ўводзіны}
\author{Яўген Пятліцкі}

\begin{document}

\maketitle

Гэты дакумэнт -- кароткія нататкі з тэорыі катэгорый. Дакумэнт
прызначаны для ўнутранага ўжытку. Тэкст сыры, зьмест, мова, стыль,
тэрміналёгія будуць зьмяняцца.

Любая зваротная сувязь вітаецца (той, хто знайшоў дакумэнт праз
пашуковік, а не атрымаў спасылку ад аўтара, можа скарыстацца
стандартнымі сродкамі гітхабу, такімі як issues).

\tableofcontents

\chapter{Катэгорыі, функтары, натуральныя пераўтварэньні}

Як казаў сп. Стэфан Банах, добры матэматык бачыць факты, выбітны --
аналёгіі паміж фактамі, а геніяльны -- аналёгіі паміж аналёгіямі.

Тэорыя катэгорый -- гэта спроба фармалізаваць аналёгіі паміж фактамі і
стварыць агульную тэорыю, у якой такія аналёгіі можна абмяркоўваць
строга.

Гістарычна тэорыя катэгорый узьнікла як інструмэнт альгебраічнай
тапалёгіі, які дазваляў фармалізаваць паняцьце ``натуральнасьці'' і
зрабіць строгімі пэўныя інтуіцыйныя разважаньні.

Ідэя тэорыі катэгорый палягае на тым, каб разгледзець цэлую клясу
аб'ектаў у іхным узаемадзеяньні паміж сабой.

Такое абагульненьне -- цалкам натуральнае: спачатку мы вучымся лічыць
канкрэтныя аб'екты фізычнага сьвету, потым разумеем (да пэўнай ступені), што такое
абстрактны цэлы лік, потым -- паралельна з абагульненьнем паняцьця
ліку -- пачынаем вывучаць канкрэтныя функцыі паміж лікамі, пасьля чаго
разглядаем усю сукупнасьць лікаў як адзін аб'ект (тое ці іншае
мноства) і пераходзім да больш абстрактнага паняцьця функцый паміж
мноствамі.

Далей зьяўляецца патрэба разгледзець мноствы, надзеленыя нейкай
структурай, геамэтрычнай альбо альгебраічнай, і функцыі паміж імі.

У нейкі момант аказваецца, што карысна вывучыць усю сукупнасьць,
напрыклад, тапалягічных прастораў альбо груп. І цалкам натуральна
ўзьнікаюць функцыі, якія пераводзяць тапалягічныя прасторы ў групы
(напрыклад, групы гаматопій). У такой сытуацыі абсягам акрэсьленьня
функцыі зьяўляецца ўся сукупнасьць (катэгорыя) тапалягічных прастораў,
а абсягам значэньняў -- катэгорыя груп.

У гэтым месцы ўважлівы чытач павінен сказаць мне ``Пачкай, а ці чуў ты
што-небудзь пра Расэлаў парадокс? Што менавіта ты называеш сукупнасьцю
ўсіх груп?''

На гэтае пытаньне я маю адказ, але троху пазьней. Пакуль уявім на
хвіліну, што мноства ўсіх мностваў нас не непакоіць (а я паабяцаю па
дарозе не будаваць аб'екты, якія ўтрымліваюць самі сябе альбо
самаадносяцца якім-небудзь іншым чынам).

\section{Азначэньне катэгорыі}

Азначэньне катэгорыі -- досыць простае, амаль трывіяльнае.

Што аб'ядноўвае ўсе сукупнасьці аб'ектаў, такіх як мноствы,
тапалягічныя прасторы, альгебры, групы, колцы?

Па-першае, ёсьць самі аб'екты.

Па-другое, ёсьць функцыі паміж імі, прычым для кожнага аб'екта існуе
тоеснае пераўтварэньне, а кампазыцыя функцыяў асацыяцыйная.

Насамрэч, гэта ўсё, што трэба каб абстрактна апісваць сукупнасьці
аб'ектаў у агульным выпадку.

\begin{definition}
  Катэгорыя $C$ -- гэта сукупнасьці аб'ектаў ${Ob C}$ (па-іншаму
  пазначаецца $C_0$) і стрэлак паміж
  імі ${Mor C}$ (па-іншаму пазначаецца $C_1$), якія  маюць наступныя
  ўласьцівасьці:

  1. Кожная стрэлка мае вызначаныя пачатак і канец (дамэн і судамэн,
  крыніцу і прызначэньне),
  якія зьяўляюцца аб'ектамі. Стрэлка $f$ з $X$ ў $Y$ запісваецца
  ${f:X\rightarrow Y}$.

  2. Для кожнай пары стрэлак ${f:X\rightarrow Y}$ і ${g:Y\rightarrow
    Z}$, вызначаная кампазыцыя ${g \circ f:X\rightarrow Z}$, калі канец першай
  супадае з пачаткам другой.

  3. Сукупнасьць стрэлак паміж аб'ектамі $X$ і $Y$ пазначаецца
  $C(X,Y)$ альбо $Hom_C(X, Y)$ і называецца гомсэтам, дзьве такія
  сукупнасьці для розных аб'ектаў ня маюць
  супольных стрэлак.

  4. Для кожнага аб'екта $X$ вызначаная адзінкавая стрэлка $1_X$, якая
  паводзіць сябе як адзінка пры кампазыцыі стрэлак, гэта значыць для
  ${f:X \rightarrow Y}$ і ${g:Z\rightarrow X}$ выконваюцца наступныя
  тоеснасьці:

  $f \circ 1_X = f$

  $1_X \circ g = g$

  5. Для кожнай тройкі стрэлак ${f: X\rightarrow Y}$, ${g:Y\rightarrow
    Z}$, ${h:Z\rightarrow W}$ выконваецца наступная тоеснасьць:

  $(h \circ g) \circ f$ = $h \circ (g \circ f)$
\end{definition}

У гэтым азначэньні мы выкарысталі больш абстрактную тэрміналёгію:
замест ``функцыі'' мы гаворым ``стрэлкі'', таму што, як мы ўбачым
пазьней, ёсьць катэгорыі, у якіх стрэлкі не зьяўляюцца функцыямі.

Па-іншаму ``стрэлка'' называецца ``марфізм''.

\begin{example}[Прыклады катэгорый]
  1. Катэгорыя мностваў $Ens$: аб'екты -- мноствы, стрэлкі -- функцыі
  паміж імі.

  2. Катэгорыя тапалягічных прастор $Top$: аб'екты -- тапалягічныя
  прасторы, стрэлкі -- непарыўныя функцыі.

  3. Катэгорыя груп $Grp$: аб'екты -- групы, стрэлкі -- гамамарфізмы.

  4. Катэгорыя часткова ўпарадкаваных мностваў $Poset$: аб'екты --
  часткова ўпарадкаваныя мноствы, стрэлкі -- манатонныя функцыі (то
  бок функцыі, якія захоўваюць частковы парадак).

  5. Любая сукупнасьць мностваў з структурай і функцыяў паміж імі,
  якія захоўваюць гэтую структуру, дае нам прыклад катэгорыі.
\end{example}

Але ці вычэрпваюцца катэгорыі толькі сукупнасьцямі мностваў з
структурай? Не, азначэньне катэгорыі дастаткова агульнае каб уключыць
іншыя выпадкі (тут я не магу не ўзгадаць, што аўтары Матэматычнай
энцыкляпэдыі, выдадзенай Тэхналёгіяй, даюць памылковае азначэньне
тэрміну ``катэгорыя'': у іх гэта толькі сукупнасьць мностваў з
структурай).

Перад тым як зьвярнуцца да больш абстрактых прыкладаў катэгорый, мы
разьбярэмся з пытаньнем іх памеру.

\section{Памер катэгорый}

Як было ўзгадана вышэй, разгляд аб'ектаў такіх як ``мноства ўсіх
мностваў'', спалучаны з клясычнай лёгікай, імгненна вядзе да
ўзьнікненьня праблем. У прынцыпе, гэта праява больш агульнай праблемы:
у клясычнай (і ў інтуіцыянісцкай, ды і ў любой кансыстэнтнай) лёгіцы
немагчыма пасьлядоўна разглядаць сытэмы з самааднясеньнем. Дастаткова
ўзгадаць парадокс цырульніка, які не патрабуе для свайго ўзьнікненьня
ніякіх мностваў.

Як можна вырашыць гэтую праблему? Першы магчымы шлях -- зьмена лёгікі
на паракансыстэнтную, то бок такую, якая дапускае для некаторых
выказваньняў магчымасьць быць адначасова праўдзівымі і непраўдзівымі,
пры гэтым не прыводзячы да трывіялізацыі ўсёй тэорыі. На гэтым шляху
на сёньняшні дзень не дасягнута істотных вынікаў, якія б пераконвалі ў
ягонай вартасьці, хаця з майго пункту гледжаньня гэты кірунак вельмі
цікавы і варты глыбейшага распрацоўваньня.

Іншы варыянт -- не зьмяняючы лёгіку, палічыць, што Расэлаў парадокс
даказвае, што сукупнасьць усіх мностваў сама мноствам не
зьяўляецца. Гэта стандартнае разьвязаньне, на якім мы і спынімся.

Аднак тут узьнікае праблема: калі мы хочам разгледзець сукупнасьць
усіх мностваў, то мы ня можам зрабіць гэтага, застаючыся ў межах
тэорыі, у якой ёсьць толькі мноствы.

Існуе некалькі разьвязаньняў праблемы, кожнае зь якіх прыводзіць да
ўвядзеньне аб'ектаў большых за мноствы -- клясаў.

Мы спынімся на Гротэндыкавых унівэрсумах.

Ідэя ўнівэрсуму палягае на тым, каб у якасьці сукупнасьці ўсіх
мностваў выбраць мноства настолькі вялікае, каб у ім мы маглі займацца
патрэбнай нам матэматыкай. Як мінімум нам неабходна быць у стане
ажыцьцяўляць стандартныя тэарэтыка-множнасныя канструкцыі такія як
утварэньне мноства ўсіх падмностваў зададзенага мноства.

Такім чынам, працуючы ўнутры нейкай тэорыі мностваў (напрыклад, ZFC),
мы выбіраем пэўную сукупнасьць мностваў, элемэнты якой будзем называць
малымі мноствамі. Сукупнасьць усіх малых мностваў сама
малым мноствам ня будзе, а будзе мноствам вялікім, альбо іншымі
словамі -- уласнай клясай.

Усе ``звычайныя'' матэматычныя канструкцыі мы будзем ажыцьцяўляць з
малымі мноствамі. Напрыклад, на іх мы будзем задаваць структуры
груп, тапалягічных прастораў альбо колцаў.

У выніку сукупнасьць усіх груп, хаця ня будзе сама малым мноствам,
аднак будзе аб'ектам, які мы можам законна разглядаць у нашай тэорыі.

Канструкцыі тэорыі катэгорый (прынамсі, калі мы хочам разглядаць
вялікія катэгорыі, такія як катэгорыя ўсіх малых мностваў)
запатрабуюць разгляду вялікіх падмностваў унівэрсуму, альбо ўсяго
ўнівэрсуму. Больш за тое, некаторыя канструкцыі запатрабуюць цэлай
герархіі ўнівэрсумаў, у якой кожны наступны ўтрымлівае папярэднія.

На нашым узроўні строгасьці -- улічваючы, што пытаньні падставаў
матэматыкі не знаходзяцца ў нашым фокусе -- хапіла б выкладзенай вышэй
заўвагі пра магчымасьць строга разглядаць сукупнасьці ўсіх мностваў і
падобныя. Але для паўніны карціны я прывяду адпаведныя азначэньні.


\begin{definition}
  Гротэндыкаў унівэрсум -- гэта мноства $U$, якое мае наступныя
  ўласьцівасьці:

  1. Для ўсіх $u \in U$ і $t \in u$ мы маем $t \in U$.

  2. Для ўсіх $u \in U$ мы маем $\mathcal{P}(u) \in U$, дзе
  $\mathcal{P}(u)$ -- сукупнасьць усіх падмностваў мноства $u$.

  3. $\mathbb{N} \in U$.

  4. Для ўсіх $I \in U$ і функцый $u: I \rightarrow U$ мы маем
  $\underset{i \in I}{\cup}u_i \in U$.
\end{definition}

Нескладана давесьці, што выконваючы звыклыя тэарэтыка-множныя апэрацыі
над элемэнтамі Гротэндыкава ўнівэрсуму, мы не выходзім за ягоныя межы.

У якасьці прыкладу дакажам, што аб'яднаньне малых мностваў -- малое
мноства (доказ ніжэй не дастаткова строгі, паколькі мы ня выбралі
аксыяматыку тэорыі мностваў, але давядзецца паверыць мне на слова, што
яго можна зрабіць поўнасьцю строгім).

\begin{example}
  З $a \in U$ і $b \in U$ вынікае, што $a \cup b \in U$.

  \begin{proof}
    З аксыёмы 3 маем $\mathbb{N} \in U$. $\emptyset \in \mathbb{N}$,
    значыць з аксыёмы 1 $\emptyset \in U$. $\mathcal{P}(\emptyset) \in
    U$ як вынікае з аксыёмы 2. Значыць, у $U$ ёсьць мноства з двух
    элемэнтаў, назавем яго $2$. Задамо функцыю $f: 2 \rightarrow U$
    такім чынам, што адзін элемэнт $2$ яна пераводзіць у $a$, а другі
    ў $b$. Паводле аксыёмы 4 $a \cup b \in U$.
  \end{proof}
\end{example}

Падобным няхітрым чынам мы можам ня толькі паказаць, што можам
займацца звычайнымі тэарэтыка-множнымі пабудовамі ў Гротэндыкавым
унівэрсуме, але і тое, што Гротэндыкаў унівэрсум сам па сабе ёсьць
мадэльлю ZFC (прынамсі, калі мы карыстаемся пры яго пабудове ZFC).

Аёй, існаваньне ўнівэрсуму -- вельмі моцная рэч, даказаць яго ў межах
ZFC не атрымаецца.

Што рабіць? Дадамо да ZFC новую аксыёму: аксыёму ўнівэрсуму.

\begin{axiom}
  Аксыёма ўнівэрсуму. Для любога мноства $s$ існуе ўнівэрсум $U$, які яго ўтрымлівае.
\end{axiom}

Падсумоўваючы, праблему катэгорыі ўсіх мностваў (і падобных вялікіх
катэгорый) мы разьвязалі ўвядзеньнем герархіі ўнівэрсумаў. У звычайнай
матэматыцы мы працуем толькі з малымі мноствамі (элемэнтамі
зафіксаванага ўнівэрсуму $U$, які сам па сабе служыць мадэльлю $ZFC$),
усе мноствы, якія ня
ёсьць малымі мы называем вялікімі мноствамі альбо ўласнымі клясамі. Пры
патрэбе, калі нашыя катэгорныя канструкцыі ня будуць зьмяшчацца ў
адным унівэрсуме, і нам спатрэбяцца яшчэ большыя мноствы, мы проста
возьмем большы ўнівэрсум. Што ён існуе, нам гарантуе аксыёма
ўнівэрсуму, якую мы дадалі да $ZFC$.

\begin{definition}
  Катэгорыя называецца малой, калі сукупнасьць яе стрэлак -- малое
  мноства, іначай катэгорыя называецца вялікай. Катэгорыя $C$ называецца
  лякальна малой, калі для любых двух яе аб'ектаў $a$ і $b$, $C(a, b)$
  -- малое мноства.
\end{definition}

Пакуль мы сустракаліся толькі зь вялікімі, але лякальна малымі катэгорыямі. Як жа выглядаюць
малыя катэгорыі?

\begin{example}
  1. Катэгорыя $0$ без аб'ектаў.

  2. Катэгорыя $1$ з адным аб'ектам і адной стрэлкай -- $1_1$.

  3. Маноід, то бок мноства, на элемэнтах якога зададзеная бінарная
  асацыяцыйная апэрацыя і ёсьць адзінкавы элемэнт. Гэтая катэгорыя мае
  толькі адзін -- фармальны -- аб'ект, а мноста яе стрэлак -- гэта
  мноства элемэнтаў маноіду.

  4. Група -- як прыклад маноіду.

  5. Мноства зь перадпарадкам. Аб'екты катэгорыі -- гэта
  элемэнты мноства, а стрэлка паміж двума элемэнтамі існуе толькі
  тады, калі паміж імі існуюць адносіны парадку.

  6. Як прыклад перадпарадку -- мноства падмностваў $\mathcal{P}(S)$
  зададзенага
  мноства $S$, упарадкаванае паводле ўключэньня.

  7. Іншы прыклад перадпарадку: мноства ўсіх адкрытых падмностваў
  $\mathcal{O}(T)$ тапалягічнай прасторы $T$, упарадкаванае паводле
  ўключэньня. Аналягічна, мноства замкнёных падмностваў
  $\mathcal{C}(T)$.

  8. Звычайнае малое мноства $S$ -- гэта прыклад катэгорыі з трывіяльнай
  структурай (дыскрэтнай катэгорыі). Яе аб'екты -- гэта элемэнты
  мноства, а стрэлкі -- адзінкавыя стрэлкі, фармальна вызначаныя для
  кожнага элемэнту. Гэта вельмі важны прыклад, які паказвае сувязь
  паняцьцяў мноства і катэгорыі. Насамрэчы, існуе некалькі герархіяў
  катэгорападобных аб'ектаў, у якіх мноствы і катэгорыі -- адны з
  найніжэйшых ступеняў.

  9. Мноства сьпісаў як прыклад маноіду. Няхай зададзенае мноства
  сымбаляў
  $\mathcal{L}$, сьпісам над $\mathcal{L}$ называецца ўпарадкаваная
  пасьлядоўнасьць сымбаляў з $\mathcal{L}$. Мноства сьпісаў над
  $\mathcal{L}$ -- гэта
  маноід адносна апэрацыі канкатэнацыі, адзінкавы элемэнт -- пусты
  сьпіс. Мноства сьпісаў -- гэта прыклад гэтак званай свабоднай
  канструкцыі, па-іншаму яно называецца свабодны маноід. Свабодны
  маноід можна задаць на любым мностве.
\end{example}

Гэтыя прыклады паказваюць нам, што катэгорыя -- не канечне сукупнасьць
мностваў з структурай і стрэлак паміж імі. Троху пазьней мы
фармалізуем паняцьці абстрактнай і канкрэтнай катэгорый. Нефармальна,
канкрэтная катэгорыя -- гэта сукупнасьць структураваных мностваў, а
абстрактная -- любая іншая катэгорыя.

Усе катэгорыі, якія мы разглядалі, -- лякальна малыя. Лякальна вялікія
катэгорыі сустрэнуцца нам пазьней.

\section{Віды марфізмаў}

На катэгорнай мове можна апісваць шмат якія знаёмыя нам зьявы
аднастайным спосабам. Мы пачнем з простай клясыфікацыі марфізмаў.

Нам ужо знаёмыя адзінкавыя марфізмы, якія ня робяць нічога.

\begin{claim}
  У катэгорыі $C$ для кожнага аб'екта $X$ існуе толькі адзін тоесны
  марфізм.
  \begin{proof}
    Дапусьцім, што акрамя $1_X$ існуе іншы марфізм $g: X \rightarrow X$
    такі, што для любых $Y, Z \in ObC$ і любых $f:Y \rightarrow X$ і
    $h:X \rightarrow Z$ выконваецца $g \circ f = f$ і $h \circ g = h$,
    тады мы маем $g = g \circ 1_X = 1_X$.
  \end{proof}
\end{claim}

Прыведзены доказ адразу даказвае ўнікальнасьць адзінкі ў групе,
маноідзе і іншых структурах, якія можна разглядаць як катэгорыю.

Няхай гэты доказ і трывіяльны, але ён ілюструе як з дапамогай паняцьця
катэгорыі мы за адзін раз даказалі сьцьверджаньне для розных
структур.

Ці можам мы ахарактарызаваць марфізмы, якія злучаюць розныя аб'екты,
але пры гэтым поўнасьцю захоўваюць іхную структуру?

\begin{definition}
  Ізамарфізм $f: X \rightarrow Y$ -- гэта марфізм, які мае зваротны,
  то бок існуе такі марфізм $g: Y \rightarrow X$, што $fg = 1_Y$ і $gf
  = 1_X$.
\end{definition}

Як лёгка праверыць, гэтае азначэньне дае чаканы вынік.

\begin{example}
  1. Для катэгорыі мностваў $Ens$ ізамарфізм -- гэта біекцыя.

  2. Для катэгорыі груп $Grp$ -- біекцыйны гомамарфізм.

  3. Для катэгорыі тапалягічных прастор $Top$ -- гамэамарфізм.
\end{example}

Такім чынам, група -- гэта маноід, у якім усе марфізмы --
ізамарфізмы. Ёсьць катэгорнае абагульненьне паняцьця групы, якое
абагульняе яго такім самым чынам, якім паняцьце катэгорыі абагульняе
паняцьце маноіда.

\begin{definition}
  Групоід -- гэта катэгорыя, усе марфізмы якой -- ізамарфізмы.
\end{definition}

\begin{example}
  Фундамэнтальны групоід тапалягічнай прасторы -- гэта катэгорыя,
  элемэнтамі якой ёсьць пункты прасторы, а стрэлкамі -- клясы
  гаматопіі шляхоў, якія захоўваюць канцавыя пункты.
\end{example}

З дапамогай ізамарфізмаў мы можам увесьці паняцьце ізаморфных
аб'ектаў.

\begin{definition}
  Аб'екты $X, Y \in C$ ізаморфныя, калі існуе ізамарфізм $f: X
  \rightarrow Y$
\end{definition}

Ізаморфнасьць аб'ектаў азначае, што па сутнасьці з пункту гледжаньня
катэгорыі $C$ яны не адрозныя.

\begin{example}
  1. У катэгорыі $Ens$ любыя два мноствы той самай магутнасьці
  ізаморфныя.

  2. У катэгорыі $Top$ гамэаморфныя прасторы -- ізаморфныя.
\end{example}

Натуральна, паняцьце ізаморфнасьці залежыць ад таго, якую структуру мы
вывучаем. Так, любыя дзьве тапалягічныя прасторы аднолькавай
магутнасьці ізаморфныя як мноствы, але не канечне ізаморфныя як
тапалягічныя прасторы (простая і плоскасьць -- той самай магутнасьці,
аднак не гамэаморфныя і, значыць, не ізаморфныя як тапалягічныя прасторы).

З абагульненьнем біекцыі мы разабраліся, а што зь сюр'екцыяй і
ін'екцыяй?

\begin{definition}
  Монамарфізм -- гэта марфізм, скарачальны зьлева, то бок такі $f: X
  \rightarrow Y$, што для любых $g, h: Z \rightarrow X$ з $fg = fh$
  вынікае $g = h$.
\end{definition}

\begin{definition}
  Эпімарфізм -- гэта марфізм, скарачальны справа, то бок такі $f: X
  \rightarrow Y$, што для любых $g, h: Y \rightarrow Z$ з $gf = hf$
  вынікае $g = h$.
\end{definition}

\begin{example}
  У катэгорыі $Ens$ монамарфізмы -- гэта ін'екцыі, а эпімарфізмы --
  сюр'екцыі.
\end{example}

Але монамарфізмы не заўжды ін'екцыі, а эпімарфізмы -- сюр'екцыі.

\begin{example}
  Разгледзім катэгорыю $Haus$ Гаўсдарфавых тапалягічных
  прастор. Укладаньне $\mathbb{Q} \hookrightarrow \mathbb{R}$ --
  эпімарфізм, але не сюр'екцыя.
\end{example}

\section{Дуальнасьць, пачатковыя і канчатковыя аб'екты}

На прыкладзе мона- і эпімарфізму мы ўбачылі пару азначэньняў, якія
паўтараюцца практычна даслоўна. Ці можна гэтую амаль даслоўнасьць неяк
фармалізаваць? Можна! Калі мы ўважліва прыгледзімся да адпаведных
азначэньняў, то ўбачым, што яны адрозьніваюцца толькі парадкам
кампазыцыі, то бок, па сутнасьці, толькі кірункам стрэлак.

\begin{definition}
  Для кожнай катэгорыі $C$ існуе супрацьлеглая, альбо дуальная,
  катэгорыя $C^{op}$, вызначаная наступным чынам

  1. $C^{op}$ мае тыя самыя аб'екты, што й $C$.

  2. Для кожнага марфізму $f: X \rightarrow Y$ з $C$ вызначаны марфізм
  $f^{op}: Y \rightarrow X$ з $C^{op}$, іншых марфізмаў у $C^{op}$ няма.

  3. Для кожнага аб'екта $X$ адзінкавая стрэлка -- гэта $1_X^{op}$.

  4. Кампазыцыя вызначаная ў зваротным кірунку, то бок $(fg)^{op}$ =
  $g^{op}f^{op}$, як лёгка спраўдзіць, для так вызначанай кампазыцыі
  выконваюцца правілы кампазыцыі з азначэньня катэгорыі.
\end{definition}

\begin{example}
  Для перадпарадку, разгледжанага як катэгорыя, дуальная катэгорыя --
  гэта супрацьлеглы перадпарадак.
\end{example}

Такім чынам мы бачым, што монамарфізм у катэгорыі $C$ будзе
эпімарфізмам у катэгорыі $C^{op}$. Падобныя паняцьці называюць
супрацьлеглымі альбо дуальнымі. Можна было б даць адно азначэньне,
напрыклад, монамарфізму і сказаць ``эпімарфізм -- паняцьце, дуальнае
монамарфізму'', пакінуўшы чытачу працу зьмяніць кірунак стрэлак.

Ашчаджэньне на азначэньнях -- гэта ня ўсё, што дае нам
дуальнасьць. Калі мы даказваем нейкую катэгорную тэарэму, мы
аўтаматычна атрымліваем дуальны вынік, паколькі паняцьце дуальнасьці
чыста сынтаксычнае і зводзіцца да зьмены кірунку стрэлак у азначэньнях
і доказах.

З дапамогай катэгорных азначэньняў мы можам характарызаваць ня толькі
марфізмы, але і аб'екты (сказанае -- трывіяльнасьць, нагадаю, што
кожнаму аб'екту $X$ адпавядае марфізм $1_X$ і тэорыю катэгорый можна
сфармуляваць наагул не карыстаючыся аб'ектамі, іхную ролю цудоўна
выконваюць адзінкавыя стрэлкі).

\begin{definition}
  Пачатковы аб'ект катэгорыі $C$ -- гэта аб'ект, зь якога ў любы
  аб'ект $X \in C$ ідзе роўна адна стрэлка.
\end{definition}

\begin{example}
  У катэгорыі $Ens$ пачатковы аб'ект -- гэта пустое мноства.
\end{example}

\begin{definition}
  Канчатковы аб'ект -- гэта аб'ект, дуальны да пачатковага, то бок
  такі, у які з кожнага аб'екта ідзе роўна адна стрэлка.
\end{definition}

\begin{example}
  У катэгорыі $Ens$ канчатковы аб'ект -- гэта мноства з адным элемэнтам.
\end{example}

Наколькі ўнікальныя пачатковы і канчатковы аб'екты? З прыкладаў мы
бачым, што ў катэгорыі $Ens$ пачатковы аб'ект -- адзін, а канчатковых
-- бясконца многа, прычым усе яны ізаморфныя.

\begin{theorem}
  Канчатковы аб'ект вызначаны з дакладнасьцю да ўнікальнага
  ізамарфізму.
  \begin{proof}
    Разгледзім канчатковыя аб'екты $X, Y \in C$.

    Паводле азначэньня з
    аб'екта $X$ у аб'екты $X$ і $Y$, і з $Y$ у $X$ існуе роўна па адным
    марфізьме. Марфізм з $X$ у $X$ -- гэта $1_X$. Назавем іншыя два
    марфізмы $f: X \rightarrow Y$ і $g:Y \rightarrow X$. Іхная
    кампазыцыя $gf: X \rightarrow X$ -- марфізм з $X$ у $X$, але такі
    марфізм па азначэньні толькі адзін -- адзначаны раней
    $1_X$. Значыць, $gf = 1_X$.

    Аналягічна, $fg = 1_Y$, а $X$ і $Y$ -- ізаморфныя. Прычым па
    азначэньні канчатковага аб'екта ізамарфізм толькі адзін.
  \end{proof}
\end{theorem}

Дуальна мы маем аналягічны вынік для пачатковага аб'екта.

Ці можа той самы аб'ект быць адначасова канчатковым і пачатковым? У
катэгорыі $Ens$ -- не, але ў цэлым -- так.

\begin{definition}
  Нулявы аб'ект -- гэта аб'ект, які адначасова пачатковы і канчатковы.
\end{definition}

\begin{example}
  У катэгорыі $Grp$ ёсьць нулявы аб'ект. Гэта трывіяльная група.
\end{example}

\section{Функтары}

Тэорыя катэгорый вучыць нас гледзячы на любую структуру пытаць, як яна
суадносіцца зь іншымі падобнымі структурамі, якія апэрацыі можна над
імі ажыцьцяўляць?

А што з самімі катэгорыямі, як выглядаюць функцыі паміж імі?

Відавочна, мы мусім вызначыць дзеяньне падобнай функцыі на аб'ектах і
на стрэлках. Акрамя таго мы мусім захаваць структуру, гэта значыць,
адзінкавы стрэлкі і кампазыцыю. Гэтага дастаткова.

\begin{definition}
  Функтар паміж катэгорыямі $A$ і $B$ -- гэта сукупнасьць функцыі
  $F:Ob A \rightarrow Ob B$ і для кожнай пары адпаведных гомсэтаў
  функцый $F_{XY}: A(X, Y) \rightarrow B(X, Y)$, такіх, што

  1. Для кожнага $X \in A$ выконваецца $F_{XX}(1_X) = 1_{F(X)}$.

  2. Для кожнай кампазавальнай пары $f, g \in Mor C$ выконваецца
  $F(gf) = F(g)F(f)$. Тут мы ня сталі пісаць індэксы ў $F$, як
  звычайна і робяць.
\end{definition}

Увага на тэму памеру: у гэтым азначэньні функцыі ў агульным
дзейнічаюць паміж клясамі.

Паняцьце функтару надзвычай багатае.

\begin{example}
  1. На кожнай катэгорыі можна задаць адзінкавы (тоесны, трывіяльны) функтар, які ня робіць
  нічога, пераводзячы аб'екты і стрэлкі ў самых сябе.

  2. Функтар забыцьця. Для кожнай катэгорыі структураваных мностваў
  вызначым функтар, які ставіць у адпаведнасьць структуры мноства, на
  якой яна зададзеная (будзем называць такое мноства
  падсподным). Напрыклад, $U: Top \rightarrow Ens$ ставіць у
  адпаведнасьць тапалягічнай прасторы мноства яе пунктаў, а $U: Grp
  \rightarrow Ens$ ставіць у адпаведнасьць групе мноства яе
  элемэнтаў. Марфізмы паміж структурамі функтар забыцьця пераводзіць у
  адпаведныя функцыі паміж мноствамі.

  3. Функтар свабоды з катэгорыі мностваў у катэгорыю груп $U: Ens
  \rightarrow Grp$ ставіць у адпаведнасьць кожнаму мноству свабодную
  групу на гэтым мностве, марфізмы вызначаюцца адпаведна (мы яшчэ
  вернемся да гэтага пытаньня пры абмеркаваньні ўнівэрсальных
  уласьцівасьцяў). Аналягічны функтар існуе для любой свабоднай
  канструкцыі, напрыклад, для свабоднага маноіду.

  4. Рэпрэзэнтацыя групы на вэктарнай прасторы. Разгледзім групу $G$
  як катэгорыю. Разгледзім таксама катэгорыю вэктарных прастор над
  полем $\mathbb{R}$ $Vec_{\mathbb{R}}$. Функтар $F: G \rightarrow
  Vec_{\mathbb{R}}$ задае рэпрэзэнтацыю групы $G$. Сапраўды, $V = F(G) \in
  Vec_{\mathbb{R}}$ -- канкрэтная вэктарная прастора. $F(1) = 1_{F(G)}
  = 1_V$. Для пары элемэнтаў групы $f, g \in G$ маем $F(gf) =
  F(g)F(f)$.

  5. Аналягічна задаецца дзеяньне групы на мностве, і наагул на любым
  аб'екце нейкай катэгорыі.

  6. Функтар паміж двума перадпарадкамі -- гэта манатонная функцыя, то
  бок функцыя, якая захоўвае перадпарадак.

  7. Функтар паміж дзьвюма групамі -- гэта гомамарфізм груп.
\end{example}

З дапамогай функтараў мы можам вызначыць ізаморфныя катэгорыі,
аналягічна таму як мы вызначалі ізаморфныя аб'екты з дапамогай стрэлак.

\begin{definition}
  Катэгорыі $C$ і $D$ называюцца ізаморфнымі, калі існуе пара
  функтараў $F: C \rightarrow D, G: D \rightarrow D$ такіх, што $GF =
  1_C$ і $FG = 1_D$.
\end{definition}

Паняцьце ізамарфізму катэгорый ня надта карыснае, каб фармалізаваць,
калі дзьве катэгорыя па сутнасьці не адрозьніваюцца, патрэбнае больш
агульнае паняцьце.

У некаторых выпадках карысна разглядаць функтары, якія дзейнічаюць не
з катэгорыі, а з дуальнай ёй.

\begin{definition}
  Функтар, які дзейнічае з дуальнай катэгорыі, называецца
  контраварыянтным (адносна $C$): $F: C^{op} \rightarrow D$.

  Звычайны функтар называцца каварыянтным.
\end{definition}

Як лёгка заўважыць, контраварыянтны функтар зьмяняе кірунак
кампазыцыі.

\begin{example}
  1. Існуе функтар з катэгорыі тапалягічных прастор у катэгорыю часткова
  ўпарадкаваных мностваў $\mathcal{O}: Top \rightarrow Poset$, які
  кожнай тапалягічнай прасторы ставіць у адпаведнасьць мноства яе
  адкрытых падмностваў, упарадкаваных паводле ўключэньня, а кожнай
  непарыўнай функцыі $f: X \rightarrow Y$ ставіць у адпаведнасьць
  функцыю мностваў $f^{-1}: \mathcal{O}(Y) \rightarrow
  \mathcal{O}(X)$, якая пераводзіць адкрытае мноства ў ягоны правобраз
  (які будзе адкрытым, таму што $f$ -- непарыўная).

  2. Аналягічна, функтар $\mathcal{C}: Top \rightarrow Poset$
  пераводзіць тапалягічную прастору ў мноства яе замкнёных
  падмностваў.

  3. Для малой катэгорыі $C$ функтар $F: C^{op} \rightarrow Ens$
  называецца перадпучком.

  4. Перадпучок на тапалягічнай прасторы $T$ -- гэта функтар $F:
  \mathcal{O}(T)^{op} \rightarrow Ens$. Прыклад такога функтара --
  перадпучок непарыўных рэчаісных функцыяў, які ставіць у
  адпаведнасьць кожнаму $U \in \mathcal{O}(C)$ мноства непарыўных
  рэчаісназначных функцыяў $f: U \rightarrow \mathbb{R}$, а на
  стрэлках дзейнічае звужэньнем (то бок для $U, V \in \mathcal{O}(C)$,
  такіх, што $U \subseteq V$, і $f \in V$ $F(U \subseteq V)(f) =
  \restr{f}{U}$).
\end{example}

\begin{example}
  Функтар, які ставіць у адпаведнасьць кожнай вэктарнай прасторы
  дуальную --
  контраварыянтны.

  Назавем наш функтар $(-)^*: Vec_{\mathbb{R}}^{op} \rightarrow Vec_{\mathbb{R}}$.

  Вэктарную прастору $V$ ён пераводзіць у $V^*$.

  Элемэнт прасторы $V^*$ -- гэта лінейны функцыянал, то бок лінейная
  функцыя з прасторы $V$ у поле, над якім зададзеная вэктарная прастора.

  Марфізм $f: V \rightarrow W$ наш функтар пераводзіць у марфізм $f^*:
  W^* \rightarrow V^*$, які дзейнічае прадкампазыцыяй на элемэнтах
  $W^*$, то бок для $\omega \in W^*$ мы маем $\omega: W \rightarrow
  \mathbb{R}$ і функтар дзейнічае так:

  $(f)^*(\omega) = f^*(\omega) = \omega \circ f: V \rightarrow
  \mathbb{R}.$
\end{example}

Функтары, як і звычайныя функцыі, могуць мець некалькі аргумэнтаў. Па
кожным з аргумэнтаў функтар можа мець сваю варыянтнасьць.

Каб разгледзець функтары многіх зьменных нам спатрэбіцца паняцьце
здабытку катэгорый.

\begin{definition}
  Здабыткам катэгорый $C$ і $D$ называецца катэгорыя $C \times D$
  такая, што

  1. Яе аб'екты -- гэта ўпарадкаваныя пары аб'ектаў з $C$ і $D$: $(c,
  d) \in Ob(C \times D)$, дзе $c \in C$ і $d \in D$.

  2. Яе марфізмы -- гэта ўпарадкаваныя пары марфізмаў з $C$ і $D$.

  3. Кампазыцыя і адзінкавыя марфізмы вызначаныя пакампанэнтна.
\end{definition}

\begin{example}
  Важным прыкладам функтара з двума аргумэнтамі ёсьць гом-функтар, які
  можна задаць на любой лякальна малой катэгорыі $C$.

  $C(-, -): C^{op} \times C \rightarrow Ens$

  Гэты функтар пераводзіць пару аб'ектаў $(a, b)$ у адпаведны гомсэт,
  а на марфізмы дзейнічае прад- і пасьлякампазыцыяй.

  Для $f: a' \rightarrow a$, $g: b \rightarrow b'$ і $h \in C(a, b)$
  маем

  $C(f, g): C(a, b) \rightarrow C(a', b'),$

  $C(f, g)(h) = g \circ h \circ f.$
\end{example}

\begin{example}
  У гом-функтары з двума аргумэнтамі можна зафіксаваць значэньне аднаго
  з аб'ектных аргумэнтаў і атрымаць каварыянтны і контраварыянтны
  гом-функтары $C(c, -)$ і $C(-, c)$.
\end{example}

Па-іншаму гом-функтары называюцца рэпрэзэнтаванымі функтарамі. У
выпадку аднааргумэнтных гом-функтараў гавораць, што $C(c, -)$ і $C(-,
c)$ рэпрэзэнтаваныя аб'ектам $c$.

\section{Падкатэгорыі}

Разьбіваць аб'екты вывучэньня на часткі бывае надзвычай карысна,
фізыкі не дадуць падмануць. Таксама як мы вызначаем падмноства,
падпрастору, падгрупу альбо падальгебру, можна вызначыць і
падкатэгорыю.

\begin{definition}
  Падкатэгорыя катэгорыі $C$ -- гэта такая катэгорыя $D$, што яе
  сукупнасьці аб'ектаў і марфізмаў -- часткі сукупнасьцяў аб'ектаў
  і марфізмаў катэгорыі $C$ ($D_0 \subset C_0$ і $D_1 \subset
  C_1$). Прычым калі $X \in D_0$, то $1_X \in D_1$, а калі $f, g \in
  D_1$, то $f \circ g \in D_1$ калі $f \circ g$ існуе ў $C$.
\end{definition}

\begin{example}
  1. Катэгорыя концых мностваў $Fin$ -- падкатэгорыя катэгорыі
  мностваў $Ens$.

  2. Катэгорыя Абэлевых груп $Ab$ -- падкатэгорыі катэгорыі груп
  $Grp$.

  3. Катэгорыя Гаўсдарфавых тапалягічных прастор $Haus$ --
  падкатэгорыя катэгорыі тапалягічных прастор $Top$.
\end{example}

\section{Натуральныя пераўтварэньні}

Фармалізацыя паняцьця натуральнасьці -- гэта тое, дзеля чаго
першапачаткова тэорыя катэгорый была створаная.

У матэматыцы мы часта сустракаем нейкія канструкцыі, якія выглядаюць
натуральнымі ў тым сэнсе, што не патрабуюць адвольнага выбару.

Напрыклад, як мы ведаем, усе концыя вэктарныя прасторы над дадзеным
полем той самай разьмернасьці -- ізаморфныя. Аднак, калі мы
прыгледзімся да гэтага пытаньня бліжэй, то заўважым, што ізамарфізм
можа быць розных тыпаў.

Так, калі мы возьмем падвойнаспалучаную прастору, то ізамарфізм паміж
$V$ і $V^{**}$ можна прад'явіць, не ажыцьцяўляючы адвольнага выбару
базісу.

Менавіта, элемэнт прасторы $V$ -- гэта вэктар $v \in V$.

Элемэнт прасторы $V^{**}$ -- гэта функцыя $\nu: V^* \rightarrow
\mathbb{K}$, якая пераводзіць элемэнт прасторы $V^*$ у элемэнт поля
$\mathbb{K}$, над якім гэтая прастора зададзеная.

У сваю чаргу, элемэнт прасторы $V^*$ -- гэта функцыя $\omega: V
\rightarrow \mathbb{K}$.

Як мы можа атрымаць з вэктара $v$ функцыю $\nu$?

А вельмі проста, нашая функцыя будзе дзейнічаць так: $\nu(\omega) =
\omega(v)$. То бок, мы проста пераводзім функцыю $\omega$ ў лік,
прымяняючы яе да зададзенага вэктара.

Вы заўважылі ў працэсе нейкі выбар базісу ці іншую адвольнасьць? Я
не. Падобныя адпаведнасьці называюцца натуральнымі.

А зараз спытаем сябе, ці можам мы падобным чынам пабудаваць ізамарфізм
паміж прасторамі $V$ і $V^*$? Не, ня можам. Як бы мы ні спрабавалі,
нам спатрэбіцца выбраць нейкі базіс, адпаведнасьць ня мае натуральнага
характару.

Паспрабуем фармалізаваць намацанае намі паняцьце. Па-першае, мы маем
эндафунктар (то бок, функтар, які дзейнічае з катэгорыі ў яе ж)
$(-)^{**}: Vec_{\mathbb{K}} \rightarrow Vec_{\mathbb{K}}$. Па-другое,
мы выкарыстоўвалі фунцыю эвалюацыі (вылічэньня) $ev$,
якая бярэ, і кожнаму элемэнту $v \in V$ ставіць у адпаведнасьць
функцыю $ev_v$, якая дзейнічае на прасторы $V^*$, прымяняючы элемэнты
гэтай прасторы да вэктара $v$.

Спачатку паглядзім, як дзейнічае функтар $(-)**$. Прастору $V$ ён
пераводзіць у $V^{**}$, а стрэлку $f: V \rightarrow W$ у стрэлку
$f^{**}: V^{**} \rightarrow W^{**}$, якая дзейнічае на элемэнтах
$V^{**}$ такім чынам ($\nu \in V^{**}$, $\omega \in W^{*}$):

$(f^{**}(\nu))(\omega) = \nu(\omega \circ f)$.

Як гэта суадносіцца зь дзеяньнем $ev$? Паспрабуем намаляваць дыяграму,
бо функцый яўна стала замнога, каб апісваць іхнае дзеяньне словамі
альбо формуламі.

\begin{tikzcd}
  V \arrow[r, "ev"] \arrow[d, "f"] & V^{**} \arrow[d, "f^{**}"] \\
  W \arrow[r, "ev"] & W^{**}
\end{tikzcd}

Аказваецца, што у якім парадку мы б не прымянялі функцыі з дыяграмы,
мы атрымаем той сам вынік. То бок для вэктара $v \in V$ і $\omega \in W^*$ мы маем

$f^{**}(ev(v))(\omega) = ev(v)(\omega \circ f) = \omega(f(v))$

і

$ev(f(v))(\omega) = \omega(f(v))$.

Якім бы шляхам па дыяграме мы ня рухаліся, пераводзячы элемэнт з $V$ у
$W^{**}$, вынік атрымліваецца той самы. Каб апісаць падобную сытуацыю
-- вынік прымяненьня функцый не залежыць ад выбранага на дыяграме
шляху -- гавораць, што дыяграма камутуе.

Гэта значыць, што наш функтар
$(-)^{**}$ узаемадзейнічае нейкім натуральным чынам з функцыяй $ev$.

Калі мы дастаткова доўга падумаем над тым, што намалявана на нашай
карцінцы, то мы ўбачым, што справа мы бачым вынік дзеяньня функтару
$(-)^{**}$, а зьлева -- адзінкавага (альбо тоеснага) функтару.

\begin{definition}
  Няхай зададзеныя два функтары, якія дзейнічаюць між тымі самымі
  катэгорыямі $F, G: C \rightarrow D$.

  Натуральным пераўтварэньнем
  паміж гэтымі функтарамі называецца набор стрэлак у $D$ (яны
  называюцца кампанэнтамі натуральнага пераўтварэньня), для
  кожнага аб'екта $c \in Ob C$ па адной стрэлцы $\alpha_c: F(c)
  \rightarrow G(c)$, такі, што для кожнай стрэлкі з $C$ $f:c
  \rightarrow c'$ дыяграма

  \begin{tikzcd}
    F(c) \arrow[r, "\alpha_c"] \arrow[d, "F(f)"] & G(c) \arrow[d, "G(f)"] \\
    F(c') \arrow[r, "\alpha_{c'}"] & G(c')
  \end{tikzcd}

  камутуе.

\end{definition}

Дыяграма з азначэньня кажа нам, што $G(f) \circ \alpha_c = \alpha_{c'}
\circ F(f)$.

Такім чынам, $ev$ з нашага матывацыйнага прыкладу задае натуральнае
пераўтварэньне з тоеснага функтару $1_{Vec_{\mathbb{K}}}$ у функтар
$(-)^{**}$. Больш за тое, гэтае натуральнае пераўтварэньне --
натуральны ізамарфізм.

\begin{definition}
  Натуральны ізамарфізм -- гэта натуральнае пераўтварэньне, кожны
  кампанэнт якога -- ізамарфізм.
\end{definition}

\begin{example}
  1. Для кожнага функтару $F$ існуе тоеснае (адзінкавае, трывіяльнае)
  натуральнае пераўтварэньне
  $1_F$, якое, натуральна, ёсьць натуральным ізамарфізмам.

  2. Функтары адкрытых і замкнёных мностваў $\mathcal{O}, \mathcal{C}:
  Top \rightarrow Poset$ натуральна ізаморфныя. Кампанэнты
  натуральнага ізамарфізму пераводзяць кожнае мноства ў ягонае
  дапаўненьне. Камутацыйнасьць дыяграмы натуральнасьці вынікае з таго,
  што ўзяцьце дапаўненьня і апэрацыя правобраза камутуюць.

  3. Возьмем дзьве вэктарныя рэпрэзэнтацыі групы $G$: $\rho_1$ і
  $\rho_2$ на прасторы $V$. Натуральнае пераўтварэньне паміж імі --
  гэта такі лінейны апэратар $T$ на $V$, што для любога $g \in G$
  мае месца $\rho_2(g) \circ u = u \circ \rho_1(g)$. Такім чынам, у
  гэтым выпадку натуральнае пераўтварэньне -- гэта гомамарфізм
  рэпрэзэнтацый, як і варта было чакаць.
\end{example}

\section{Катэгорыі функтараў}

Натуральныя пераўтварэньні -- гэта марфізмы паміж функтарамі. Як мы
ўжо заўважылі, існуе тоеснае натуральнае пераўтварэньне. Такім чынам,
мы ў кроку ад таго, каб паглядзець на сукупнасьць функтараў з
катэгорыі $C$ у катэгорыю $D$ як на катэгорыю.

Чаго нам бракуе? Кампазыцыі натуральных пераўтварэньняў. Няхай мы маем
тры функтары $F, G, H: C
\rightarrow D$ і два натуральныя пераўтварэньні $\alpha: F \rightarrow
G$ і $\beta: G \rightarrow H$. Давайце
паспрабуем намаляваць дыяграму. Возьмем у катэгорыі $C$ аб'екты $c$ і $c'$.

\begin{tikzcd}
  F(c) \arrow[r, "\alpha_c"] \arrow[d, "F(f)"] & G(c) \arrow[r,
    "\beta_c"] \arrow[d, "G(f)"] & H(c) \arrow[d, "H(f)"] \\
  F(c') \arrow[r, "\alpha_{c'}"] & G(c') \arrow[r, "\beta_{c'}"] & H(c')
\end{tikzcd}

Мы хочам атрымаць натуральнае пераўтварэньне $\beta \circ \alpha: F
\rightarrow H$. Што, калі ў якасьці ягоных кампанэнтаў узяць
кампазыцыю кампанэнтаў пераўтварэньняў $\alpha$ і $\beta$?

Нам трэба праверыць, што ўмова натуральнасьці выкананая. Кожны з
квадратаў дыяграмы -- камутацыйны, то бок $H(f) \circ \beta_c$ =
$\beta_{c'} \circ G(f)$ і $G(f) \circ \alpha_c = \alpha_{c'} \circ
F(f)$.

Роўнасьць $(\beta_{c'} \circ \alpha_{c'}) \circ F(f) = H(f) \circ
(\beta_c \circ \alpha_c)$ відавочная.

Асацыяцыйнасьць такой кампазыцыі натуральных пераўтварэньняў вынікае з
асацыяцыйнасьці кампазыці функцый.

\begin{definition}
  Няхай зададзеныя дзьве катэгорыі $C$ і $D$. Катэгорыяй функтараў з
  $C$ у $D$ называецца катэгорыя $D^C$, аб'ектамі якой ёсьць усе
  функтары з $C$ у $D$, а стрэлкамі -- натуральныя пераўтварэньні
  паміж імі з кампазыцыяй, вызначанай як кампазыцыя функцый-кампанэнтаў.
\end{definition}

Што можна сказаць пра памер такой катэгорыі? Яна можа быць ня проста
вялікай, а надзвычай вялікай. Возьмем для прыкладу катэгорыю
мностваў $Ens$ і катэгорыю з двума аб'ектамі $2_I$, у якой два
супрацьлеглыя нетрывіяльныя марфізмы (
\begin{tikzcd}
  0 \arrow[r, "01"] & 1 \arrow[l, "10"]
\end{tikzcd}
). Каб задаць функтар з $Ens$ у $2_I$ трэба выбраць для кожнага мноства з
$Ens$ аб'ект-вобраз у $2_I$. Стрэлкі паміж мноствамі з аднаго
правобраза пераходзяць у адпаведную адзінкавую стрэлку, а стрэлкі
паміж мноствамі з розных правобразаў -- у адпаведную стрэлку паміж
аб'ектамі $2_I$. Колькі такіх функтараў існуе? Столькі, колькі
падмностваў у $Ob(Ens)$, то бок столькі, колькі падмностваў ва
ўнівэрсуме $U$. Такім чынам, падобная катэгорыя сама не зьмяшчаецца ва
ўнівэрсум $U$ і патрабуе большага ўнівэрсуму.

\begin{example}
  1. Катэгорыя функтараў з групы $G$ у катэгорыю вэктарных прастораў над
  полем $\mathbb{K}$ $Vec_{\mathbb{K}}$ -- гэта катэгорыя
  $\mathbb{K}$-вэктарных рэпрэзэнтацый групы $G$.

  2. Для малой катэгорыі $C$ катэгорыя $Ens^C$ -- гэта катэгорыя
  перадпучкоў, па-іншаму яна пазначаецца $\hat{C}$.
\end{example}

\section{Азначэньні з дапамогай дыяграмаў}

У гэтай і наступнай частках абмяркуем больш падрабязна выкарыстаньне
камутацыйных дыяграм у тэорыі катэгорый.

Нефармальна, дыяграма -- гэта набор аб'ектаў катэгорыі і марфізмаў
паміж імі. Камутацыйная дыяграма -- гэта такая дыяграма, у якой
кампазыцыі марфізмаў з пачаткам і канцом у тых самых аб'ектах не
залежаць ад выбранага шляху.

Разгледзім, напрыклад, такую камутацыйную дыяграму

\begin{tikzcd}
  X \arrow [r, "f"] \arrow[d, "h"] & Y \arrow[r, "g"]  \arrow[d, "k"] & Z \arrow[d, "l"] \\
  U \arrow[r, "m"] & V \arrow[r, "n"] & W
\end{tikzcd}

Камутацыйнасьць гэтае дыяграмы азначае, што $l \circ g \circ f = n
\circ k \circ f = n \circ m \circ h$, але таксама і $k \circ f = m
\circ h$, і гэтак далей.

Камутацыйныя дыяграмы зручна выкарыстоўваць пры аналізе суадносінаў
паміж марфізмамі і аб'ектамі ў катэгорыі, у доказ і азначэньнях.

Для прыкладу, выразім на мове камутацыйных дыяграм умову
асацыяцыйнасьці з азначэньня катэгорыі ($(h \circ g) \circ f = h
\circ (g \circ f)$).

\begin{tikzcd}
  X \arrow [r, "f"] \arrow [rd, "g \circ f"] & Y \arrow[d, "g"] \arrow [rd, "h \circ g"]\\
  & Z \arrow[r, "h"] & V
\end{tikzcd}

Справіўшыся з гэтай задачай, давайце паспрабуем даць на дыяграмнай
мове якое-небудзь знаёмае азначэньне. Напэўна, найпрасьцейшы
альгебраічны аб'ект -- гэта маноід. Як ужо ўспаміналася, маноід --
гэта мноства з зададзенай на ім бінарнай асацыяцыйнай апэрацыяй і
вылучаным адзінкавым элемэнтам.

Для заданьня бінарнай апэрацыі нам спатрэбіцца Дэкартаў здабытак
мностваў (апэрацыя задаецца на пары элемэнтаў, а вызначэньне
асацыяцыйнасьці патрабуе трох элемэнтаў), і мноства з адным
элемэнтам (назавем яго проста $1$), каб выбраць адзінку маноіда.

Такім чынам, мы выбіраем мноства $M$ (аб'ект у катэгорыі $Ens$),
функцыю $\mu: M \times M \rightarrow M$, якая будзе прадстаўляць
нашую апэрацыю, і функцыю $\eta: 1 \rightarrow M$.

Як працуе апэрацыя, ясна: яна бярэ два элемэнты і робіць зь іх
адзін. А як працуе выбар адзінкавага элемэнта? Ён бярэ адзіны
элемэнт мноства $1$ і пераводзіць яго ў фіксаваны элемэнт мноства
$M$, які мы і будзем называць адзінкай маноіду $M$.

Засталося выразіць дыяграмамі асацыяцыйнасьць апэрацыі і тое, што
адзінка сапраўды паводзіць сябе як адзінка адносна нашай апэрацыі.

\begin{example}
  Маноідам мы называем аб'ект $M$ катэгорыі $Ens$ разам з стрэлкамі
  $\mu: M \times M \rightarrow M$ і $\eta: 1 \rightarrow M$ такімі,
  што наступныя дыяграмы камутуюць:

  \begin{tikzcd}
    M \times M \times M \arrow [r, "1_M \times \mu"] \arrow [d,
      "\mu \times 1_M"] & M \times M \arrow[d, "\mu"]\\
    M \times M \arrow[r, "\mu"] & M
  \end{tikzcd}
  \begin{tikzcd}
    M \arrow [r, "\eta \times 1_M"] \arrow [dr, "1_M"]
    & M \times M \arrow[d, "\mu"] & M \arrow[l, "1_M
      \times \eta"] \arrow [dl, "1_M"] \\
    & M
  \end{tikzcd}


  Перакладаючы дыяграмы на мову формул, атрымліваем

  1. Асацыяцыйнасьць: для элемэнтаў $a, b, c \in M$ маем $\mu(a,
  \mu(b, c)) = \mu(\mu(a, b), c)$.

  2. Адзінка: для элемэнтаў $a, b \in M$ маем $\mu(\eta(1), a) = a$
  і $\mu(b, \eta(1)) = b$.
\end{example}

Азначэньні на мове дыяграм карысныя ня толькі сваёй нагляднасьцю,
але і лёгкасьцю перанясеньня з катэгорыі ў катэгорыю.

Так, для азначэньня маноіду нам спатрэбіліся толькі Дэкартаў
здабытак і мноства з адным элемэнтам, якое паводзіць сябе як
адзінка адносна Дэкартава здабытку і дазваляе выбраць адзінкавы
элемэнт у мностве.

А што, калі замест катэгорыі $Ens$ мностваў мы возьмем катэгорыю
$Ab$ Абэлевых груп?

Тут нам давядзецца троху папрацаваць. Асноўнае пытаньне, што мы хочам
лічыць здабыткам Абэлевых груп?

\section{Катэгорны здабытак}

Спачатку прааналізуем з катэгорнага пункту гледжаньня канструкцыю
Дэкартавага здабытку.

У чым сутнасьць здабытку двух мностваў, як мы яго прымяняем?

Для
мностваў $X, Y \in Ob Ens$ мы маем нейкі аб'ект $X \times Y$, з
элемэнтаў якога мы можам атрымаць ягоныя складовыя часткі -- элемэнт
$X$ і элемэнт $Y$. Фармалізуючы гэта, скажам, што існуюць праекцыі
$p_1: X \times Y \rightarrow X$ і $p_2: X \times Y \rightarrow Y$.

Калі для нейкага мноства $P$ мы маем пару функцый
$f: P \rightarrow X$, $g: P \rightarrow Y$, то існуе функцыя $u: P
\rightarrow X \times Y$, такая, што $p_1 \circ u = f$ і $p_2 \circ u =
g$. Элемэнту $p \in P$ гэтая функцыя ставіць у адпаведнасьць пару
$(f(p), g(p)) \in X \times Y$.

Існаваньне функцыі $u: P \rightarrow X \times Y$ для кожнай пары
функцый $f: P \rightarrow X$ і $g: P \rightarrow Y$ азначае, што $X
\times Y$ зьмяшчае ўсе магчымыя пары $\{(x, y) | x \in X, y \in
Y\}$. Калі $X \times Y$ не зьмяшчае ніякай іншай інфармацыі, такая
функцыя -- адзіная.

Цяпер мы гатовыя даць катэгорнае -- у тэрмінах марфізмаў -- азначэньне
здабытку, якое будзе мець сэнс ня толькі ў катэгорыі мностваў і ня
будзе вызначаць канкрэтную канструкцыю.

\begin{definition}
  Здабыткам двух аб'ектаў $X, Y$ катэгорыі $C$ называецца аб'ект $X
  \times Y$ і пара марфізмаў $p_1: X \times Y \rightarrow X$ і $p_2: X
  \times Y \rightarrow Y$ такія, што для любога $P \in C$ і пары
  марфізмаў $f: P \rightarrow X$ і $g: P \rightarrow Y$ існуе адзіны
  марфізм $u:P \rightarrow X \times Y$ такі, што $p_1 \circ u = f$ і
  $p_2 \circ u = g$. Іншымі словамі, існуе адзіны марфізм $u$ такі,
  што дыяграма ніжэй камутуе

  \begin{tikzcd}
    & P \arrow [ldd, "f"] \arrow [d, "!u"] \arrow [rdd, "g"] & \\
    & X \times Y \arrow [ld, "p_1"] \arrow [rd, "p_2"] & \\
    X & & Y
  \end{tikzcd}
\end{definition}

Ці заўжды існуе так вызначаны здабытак? Не, ёсьць катэгорыі, у якіх
здабыткі не існуюць, альбо існуюць не для ўсіх элемэнтаў.

Тое, што нам трэба вынесьці з прыведзенага прыкладу -- гэта ідэя
задаваць аб'ект не канструкцыяй, а ягонымі так званымі ўнівэрсальнымі
ўласьцівасьцямі.

Каб задаць унівэрсальную ўласьцівасьць мы спачатку выбіраем нейкую
ўласьцівасьць (як правіла гэта наяўнасьць нейкіх марфізмаў з аб'екта,
альбо ў яго), выбіраем спосаб параўнаньня аб'ектаў паміж сабой
(напрыклад, наяўнасьць марфізму паміж двума аб'ектамі), а потым
выбіраем найлепшы, унівэрсальны, аб'ект, той, які зададзеным чынам
параўноўваецца з усімі іншымі аб'ектамі катэгорыі.

Прааналізуем з гэтага пункту гледжаньня азначэньне концага
аб'екта. Уласьцівасьць -- гэта проста ўласьцівасьць быць аб'ектам
катэгорыі. Спосаб параўнаньня: наяўнасьць стрэлкі паміж двума
аб'ектамі, прычым лепшы -- той аб'ект, які служыць канцом
стрэлкі. Унівэрсальны аб'ект -- такі, у які ідзе роўна адна стрэлка з
кожнага аб'екта катэгорыі.

Для здабытку ўласьцівасьць -- гэта мець дзьве стрэлкі з аб'екта ў два
зададзеныя аб'екты $X$ і $Y$. Параўнаньне -- аналягічна як для концага
аб'екта
-- наяўнасьці стрэлкі паміж аб'ектамі, пры чым лепшы той, які ёсьць
канцом стрэлкі. З тым, што цяпер мы параўноўваем ня проста аб'екты, а
аб'екты, зь якіх ідуць дзьве стрэлкі ў $X$ і $Y$.
Унівэрсальны -- той аб'ект з парай стрэлак у $X$ і $Y$, у які ёсьць роўна
па адной стрэлцы зь любога аб'екта з парай стрэлак у $X$ і $Y$.

\section{Мультыкатэгорыі}

Перад тым, як займацца Абэлевымі групамі, разгледзім больш просты
выпадак, а менавіта вэктарныя прасторы. Няхай мы маем катэгорыю
$Vec_{\mathbb{K}}$ вэктарных прастор
над полем $\mathbb{K}$. У якасьці здабытку прастор $V, W \in
Vec_{\mathbb{K}}$ возьмем звычайны тэнзарны здабытак $V \otimes
W$. Нагадаю ягоную канструкцыю.

Пачнем з школьнай -- некатэгорнай -- канструкцыі. Нефармальна прастора
$V \otimes W$ -- гэта вэктарная прастора, якая мае ў якасьці базісу
мноства ўсіх пар $\{v \otimes w|v \in B_V, w \in B_W\}$, дзе $B_V$ і
$B_W$ -- выбраныя базісы $V$і $W$. Элемэнты гэтай
прасторы называюцца тэнзарамі.

Тэнзарны здабытак двух адвольных вэктараў задаецца празь іхны расклад
на базісы. Лёгка паказаць, што функцыя $- \otimes -: V \times W
\rightarrow V \otimes W$ ($V \times W$ -- звычайны Дэкартаў здабытак)
білінейная.

Можна даць незалежнае ад базісу азначэньне тэнзарнага здабытку як
фактар-прасторы.

У тэрмінах унівэрсальных уласьцівасьцяў тэнзарны здабытак задаецца
наступным чынам.

\begin{definition}
  Тэнзарным здабыткам двух вэктарных прастораў $V$ і $W$ называецца
  аб'ект $V \otimes W$, у які ёсьць білінейная функцыя $- \otimes -: V \times W
  \rightarrow V \otimes W$ такая, што для любой іншай білінейнай
  функцыі $b: V \times W \rightarrow U$ існуе адзіная лінейная
  функцыя $u: V \otimes W$ такая, што дыяграма ніжэй камутуе

  \begin{tikzcd}
    V \times W \arrow [r, "b"] \arrow [d, "- \otimes -"] & U\\
    V \otimes W \arrow [ru, "!u"] &
  \end{tikzcd}

  па-іншаму гавораць, што марфізм $b$ адназначна фактарызуецца пра
  марфізм $- \otimes -$.
\end{definition}

Як і для звычайнага здабытку лёгка паказаць, што такі здабытак існуе
адзін з дакладнасьцю да адзінага ізамарфізму. Лёгка паказаць таксама,
што азначэньні з дапамогай базісаў і з дапамогай фактар-прасторы
адпавядаюць гэтай унівэрсальнай уласьцівасьці і такім чынам,
эквівалентныя.

Што кепска з нашым унівэрсальным азначэньнем? Кепска тое, што яно ня
поўнасьцю катэгорнае ў тым сэнсе, што частка структуры, якую яна
выкарыстоўвае ня выражаная непасрэдна ў тэрмінах ужываных
катэгорый. Гаворка пра білінейныя функцыі. Нагадаю, што мы разглядаем
аб'екты $Vec_{\mathbb{K}}$, марфізмы паміж якімі -- лінейныя функцыі,
і раптам нам спатрэбіўся яшчэ адзін від марфізмаў, азначэньне якога
паходзіць з разгляданай прадметнай галіны -- білінейныя
функцыі. Натуральна, так можна жыць, але ці можам мы даць больш
катэгорнае азначэньне, у якім усё будзе фармалізавана катэгорнай
мовай?

Чаго нам не хапае? Нам патрэбныя полілінейныя функцыі. У звычайнай
катэгорыі мы можам задаць функцыі некалькіх аргумэнтаў, калі існуюць
здабыткі (такая функцыя -- гэта проста $f: X \times Y \rightarrow Z$),
аднак для выпадку катэгорыі $Vec_{\mathbb{K}}$ такая фнукцыя будзе
лінейнай на $X \times Y$, а не білінейнай.

Нам магло б дапамагчы ўбудаванае ў катэгорыю паняцьце функцыі
некалькіх аргумэнтаў. Катэгорныя структуры з стрэлкамі, дамэн якіх --
сьпіс
аб'ектаў, існуюць і называюцца мультыкатэгорыямі.

\begin{definition}
  Мультыкатэгорыя $C$ складаецца з наступных аб'ектаў:

  1. Сукупнасьць аб'ектаў $C_0$.

  2. Сукупнасьць мультыстрэлак (мультымарфізмаў) $C_1$.

  3. Адлюстраваньня, якое для кожнай мультыстрэлкі задае яе крыніцу
  $s: C_1 \rightarrow (C_0)*$, дзе $(C_0)*$ -- свабодны маноід на
  $C_0$ (то бок мноства концых сьпісаў на $C_0$, пусты сьпіс
  уключаны).

  4. Адлюстраваньня, якое для кожнай мультыстрэлкі задае яе
  прызначэньне $t: C_1 \rightarrow C_0$.

  Мультыстрэлка запісваецца як $f: X_1,...,X_n \rightarrow Y$.

  Сукупнасьць мультыстрэлак з $X_1, ..., X_n$ у $Y$ пазначаецца
  $C(X_1, ..., X_n;Y)$ альбо $Hom_C(X_1, ..., X_n;Y)$.

  На зададзеных вышэй аб'ектах вызначаныя наступныя апэрацыі:

  1. Адлюстраваньне $1_\_: C_0 \rightarrow C_0$ вызначае для кожнага
  $X \in C_0$ адзінкавую стрэлку $1_X$.

  2. Для кожнага набора лікаў $n, k_1, ..., k_n \in \mathbb{N}$ і
  аб'ектаў $Y, X_i, X_i^j \in C_0$ вызначаная кампазыцыя

  $\circ: C(X_1, ..., X_n; Y) \times C(X_1^1, ..., X_1^{k_1}; X_1)
  \times ... \times C(X_n^1, ..., X_n^{k_n}; X_n) \rightarrow C(X_1^1,
  ..., X_1^{k_1}, ..., X_n^1, ..., X_n^{k_n}; Y)$,

  якая запісваецца $f \circ (f_1, ..., f_n)$.

  Для так зададзенай структуры выконваюцца наступныя аксыёмы -- у
  кожным выпадку, калі адпаведныя кампазыцыі вызначаныя:

  1. Адзінкавасьць: $f \circ (1_{X_1}, ..., 1_{X_n}) = f = 1_Y \circ
  f$.

  2. Асацыяцыйнасьць: $f \circ (f_1 \circ (f_1^1, ..., f_1^{k_1}),
  ..., f_n \circ (f_n^1, ..., f_n^{k_n})) = (f \circ (f_1, ..., f_n))
  \circ (f_1^1, ..., f_1^{k_1}, ..., f_n^1, ..., f_n^{k_n})$.
\end{definition}

Запісаныя формуламі аксыёмы мультыкатэгорый выглядаюць ня надта
чытэльна (хаця тых, хто бачыў тэнзары, пужаць не павінныя).

\begin{definition}
  Стрэлка з дамэнам з аднаго аб'екта называецца унарнай.
\end{definition}

\begin{example}
  1. Мультыкатэгорыя, у якой усе стрэлкі ўнарныя -- гэта звычайная
  катэгорыя.

  2. Мультыкатэгорыя $MVec_{\mathbb{K}}$ вэктарных прастор над полем
  $K$ з полілінейнымі адлюстраваньнямі.
\end{example}

Маючы ў руках новы інструмэнт, мы можам даць натуральнае азначэньне
тэнзарнага здабытку.

\begin{definition}
  Тэнзарным здабыткам аб'ектаў $X$ і $Y$ мультыкатэгорыі $C$
  называецца такі аб'ект $X \otimes Y \in C$ разам з мультымарфізмам
  $- \otimes -: X, Y \rightarrow X \otimes Y$, што любы мультымарфізм
  $f: X, Y \rightarrow Z$ адназначна фактарызуецца праз $- \otimes -:
  X, Y \rightarrow X \otimes Y$, то бок для любога $f$ існуе адзіны
  мультымарфізм $u: X \otimes Y \rightarrow Z$ такі, што дыяграма
  ніжэй камутуе:

  \begin{tikzcd}
    X, Y \arrow [r, "f"] \arrow [d, "- \otimes -"] & Z\\
    X \otimes Y \arrow [ru, "!u"] &
  \end{tikzcd}
\end{definition}

\end{document}
